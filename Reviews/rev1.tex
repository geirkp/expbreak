\newcommand{\refpoint}[1]{\ \vspace{0.3cm}\\ {\em #1}\  \vspace{0.3cm}\\ }
Reviewer \#1:
\refpoint{This paper reports a combined laboratory and numerical modeling of a set of solitary-wave runup tests 
with mostly plunging breaking waves during the process.  Laboratory measurements were conducted in a small scale tank 
(water depth 0.2 m) with the velocity field measured using PIV at 4 locations and the runup height measured using a 
camera as well as acoustic gauges.  The experiment seems to be carefully conducted and of good quality.  The numerical
 model was validated with good agreement.  Nevertheless, the numerical model, a boundary integral model combined with a
 viscous boundary layer model, is not new - it has been published in Physics of Fluids in 2013 by the authors.}
Yes, the BIM and boundary layer models have been presented and used before, and are thus only sketchily described 
in the present paper. Still, the smaller inclination angle, in relation to the 2013 article, makes the 
computations more demanding. Hence, new tests and documentaion on accaracy are required.\\
Anyway, these models are not at the hearth of the submitted paper.
The main point of their application  is to demonstrate that the experimental setup do
provide data in good agreement with theory when the flow is regular. This supports that the irregularities
observed in the other measurements are real.   
 \refpoint{The 
experimental data, although some have been used for validation in the POF paper, are mostly new and interesting to 
researchers in the related area.}
No data from the present investigation was used in the 2013 article, or vice versa. The angle of inclination
is different and the incident waves (fig.3) have been measured anew, and even in a slightly modified manner. 
Additions are made in the introduction to clarify the relation between the 2013 paper and the present one
(lines ....  and ....).
\refpoint{ Even though this study adds values to our understanding of solitary wave runup and 
breaking, I don't see clear new findings from the study.
 I feel the study may be published as a Technical Note rather than a Technical Paper due to its limited scope and 
findings, providing the following comments are adequately addressed.}
We don't quite follow the referee here. In the paper we present a new and rather elaborate set of experiments.
Phenomena such as bubble dynamics to to plunging breakers at the shore are examined, 
the paper has normal length and our scope and findings are no more narrow or unclear than what is common in 
papers on irregular flows.
We believe that the paper should not be reduced to a technical note. 
1.The figure with the measured surface elevation (FIG 3, and  FIG 4) have been  nondimentiolized  by the water depth. Also the runup height and the shoreline positions figures FIG 5 and 6 is scaled with the water depth. The figures that show boundary layers are not dimensional, due to non-scaled behaviour layer. 


2.The number of repetitions is three (N=3). This number is chosen due to practical reasons. Between each of the run, the water needs settle, which takes approximately 45 minutes depending on wave characteristics. Our goal in this study is not to examine the turbulent structures, but to investigate how the velocities profiles changes due to different wave height and at different location on the beach.

3. The gradient magnitude images tell us where you have sharp edges in the images, and in our case this will responds to the interphase between water and air. 



4. The maximum runup height for the breaking waves for the BIM model were not defined, since the model breaks down long before maximum runup. Figure 5 is a observed shoreline from the wavetank. The maximum runup height was defined as the highest impingement of water on the beach for both the BIM and the experimental study. 

5.


6. 

7. The figures are so unsmooth due to the relatively small size of the swash 


8. In the paper by Chang and Liu, they see an bias error in PIV algoritm, which result in Psedo turbulence intensities in the non breaking waves. They sugguest that the error is related to the ration of the particle image to the pixel size in the images. This is often reffered to as peak locking in PIV. However the oscillations in Figure 12, is found by  performing PTV on the images. We would expect to get the same type of bias error with the PTV technique, but the error could by minimized by particles. 
