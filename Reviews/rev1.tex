Reviewer \#1:

1.The figure with the measured surface elevation (FIG 3, and  FIG 4) have been  nondimentiolized  by the water depth. Also the runup height and the shoreline positions figures FIG 5 and 6 is scaled with the water depth. The figures that show boundary layers are not dimensional, due to non-scaled behaviour layer. 


2.The number of repetitions is three (N=3). This number is chosen due to practical reasons. Between each of the run, the water needs settle, which takes approximately 45 minutes depending on wave characteristics. Our goal in this study is not to examine the turbulent structures, but to investigate how the velocities profiles changes due to different wave height and at different location on the beach.

3.

 The gradient magnitude images tell us where you have sharp edges in the images, and in our case this will responds to the interphase between water and air. 



4. The maximum runup height for the breaking waves for the BIM model were not defined, since the model breaks down long before maximum runup. Figure 5 is a observed shoreline from the wavetank. The maximum runup height was defined as the highest impingement of water on the beach for both the BIM and the experimental study. 

5.


6. 

7. The figures are so unsmooth due to the relatively small size of the swash 


8. In the paper by Chang and Liu, they see an bias error in PIV algoritm, which result in Psedo turbulence intensities in the non breaking waves. They sugguest that the error is related to the ration of the particle image to the pixel size in the images. This is often reffered to as peak locking in PIV. However the oscillations in Figure 12, is found by  performing PTV on the images. We would expect to get the same type of bias error with the PTV technique, but the error could by minimized by particles. 
