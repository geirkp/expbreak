
\newcommand{\refpoint}[1]{\ \vspace{0.3cm}\\ {\em #1}\  \vspace{0.3cm}\\ }
\newcommand{\todo}[1]{\ \\ {\bf To do: #1}}
Reviewer \#1:

\refpoint{This paper reports a combined laboratory and numerical modeling of a set of solitary-wave runup tests 
with mostly plunging breaking waves during the process.  Laboratory measurements were conducted in a small scale tank 
(water depth 0.2 m) with the velocity field measured using PIV at 4 locations and the runup height measured using a 
camera as well as acoustic gauges.  The experiment seems to be carefully conducted and of good quality.  The numerical
 model was validated with good agreement.  Nevertheless, the numerical model, a boundary integral model combined with a
 viscous boundary layer model, is not new - it has been published in Physics of Fluids in 2013 by the authors.}
Yes, the BIM and boundary layer models have been presented and used before, and are thus only sketchily described 
in the present paper. Still, the smaller inclination angle, in relation to the 2013 article, makes the 
computations more demanding. Hence, new tests and documentaion on accaracy are required.\\
Anyway, these models are not at the hearth of the submitted paper.
The main point of their application  is to demonstrate that the experimental setup do
provide data in good agreement with theory when the flow is regular. This supports that the irregularities
observed in the other measurements are real.   
 \refpoint{The 
experimental data, although some have been used for validation in the POF paper, are mostly new and interesting to 
researchers in the related area.}
No data from the present investigation was used in the 2013 article, or vice versa. The angle of inclination
is different and the incident waves (fig.3) have been measured anew, and even in a slightly modified manner. 
Additions are made in the introduction to clarify the relation between the 2013 paper and the present one
(lines ....  and ....).
\refpoint{ Even though this study adds values to our understanding of solitary wave runup and 
breaking, I don't see clear new findings from the study.
 I feel the study may be published as a Technical Note rather than a Technical Paper due to its limited scope and 
findings, providing the following comments are adequately addressed.}
We don't quite follow the referee here. In the paper we present a new and rather elaborate set of experiments.
Phenomena such as bubble dynamics to to plunging breakers at the shore are examined, 
the paper has normal length and our scope and findings are no more narrow or unclear than what is common in 
papers on irregular flows.
We believe that the paper should not be reduced to a technical note. 

\refpoint{1.	To make the results useful to other researchers, I suggest that the authors nondimensionalize their measurement 
results, similar to what they did in Table 2.  For example, the measured velocity may be normalized by the phase speed,
 and, similarly, the vertical quantities by the wave height and the horizontal quantities by the approximate wavelength
 (similar to what Grilli did). }

It is not clear to us exactly what paper of Grilli the referee has in mind, but we take it that he suggests a scaling
with the waterdepth $H$ as length scale and $\sqrt{gH}$ as velocity scale. This may be appropriate for quantities like outer flow velocities and run, but not at all for boundary thicknesses, for instance.
The figure with the measured surface elevation (FIG 3, and  FIG 4) have been  made dimensionless  by the water depth. Also the runup height and the shoreline positions figures FIG 5 and 6 is scaled with the water depth. The figures that show boundary layers are still given with units. 
\todo{Geir; check axis legends in changed plots}

\refpoint{2.	Line 38.  Based on Figs. 8-11, N = 3 was used in the study?  The information is not provided in the text.  
If so, it is too small if the flow is turbulent. }
The number of repetitions is three (N=3). This number is chosen due to practical reasons. Between each of the run, the water needs settle, which takes approximately 45 minutes depending on wave characteristics. 
We do not attempt to quantify turbulence . Instead we investigate the evolution of overall velocity profiles on the beach.
\todo{Fix the text}

\refpoint{3.The photos in Figs. 2 and 13-14 are difficult to read and understand. Not sure what the causes are (the original images, the image enhancing (gradient) process, and/or the pseudo color?)!}

Figure 2 is changed to a raw image from $\alpha=0.30$, where the contrast has been enhanced instead of using the scaled image from matlab. Hopefully, this black and white image will be easier to understand. The Image is also rotated with the same inclination as the beach. More information and interpretation regarding the gradient magnitude images are provided in the manuscript, and these images are also rotated.



\refpoint{4.	How was the maximum runup height defined in Fig. 5 for breaking waves? Was it defined only up to the impingement
 (after that the model cannot handle)?  If so, that is not a typical definition of wave runup height. }
4. The maximum runup height for the breaking waves for the BIM model were not defined, since the model breaks down long before maximum runup. This is explicitly stated in the start of sections 2.3 and 3.2.  The text is also amended in these locations. Figure 5 displays the observed shoreline from the wavetank. The maximum runup height is defined as the highest beach elevation water on the beach for both the BIM model (available only for the smallest amplitude) and the experiments. \todo{Add which cases are computed by BIM in 2.3, make back reference on start 0f 3.2 more explicit}

\refpoint{5.	The average deviation, sigma bar, in Table 3 varies from 0.01 to 0.1 if normalized by u = 0.4 m/s.  
This magnitude is too small to resemble the turbulent level in a turbulent flow, but too large to resemble the
 fluctuation level in a laminar flow.  Even though the authors mentioned that the flow was laminar and transitional, 
the results in the table need more elaboration to interpret.
 }
Our point is to quantify the apparent poorer repeatability for case 50 than for ... This poorer repeatability then
points to a transitional flow due to the breaking or local instabilities. We are at present not able to
distinguish firmly.  
\todo{Add some more lines close to 176}


\refpoint{6.	The flow is unlikely in the turbulent regime due to the small physical scale.  It is perfectly fine to conduct
 such laboratory-scale experiment so advanced techniques such as PIV can be employed for detailed flow measurements.  
However, the authors should also address the limitation of using such a small scale setting and its results in practical
 applications. }
Most experiment are performed on a small scale and some effects are probably less scale dependent than others.
We definetely agree with the referee that this should be emphasized as a rule.
Possible limitations due to the small scale is now discussed in the discussion section.
\todo{Discuss it}
\refpoint{ 7.	Figs. 10-11.  Why are the mean velocities so unsmooth?  If the flow is turbulent or transitional, presenting the
 mean velocity based on 3 repetitions is too meaningful.
 }
\todo{Is it the seeding or what ? What is stated in the text. }The figures are so unsmooth due to the relatively small size of the swash 


\refpoint{8.	 Fig. 12.  Are these "oscillations," as the authors described, realistic?  I would think such oscillations may 
represent the encounter of eddies at and beyond the breaking.  Nonetheless, the measurements were taken prior to the 
breaking while the occurrence and sequence of eddies seem to be too regular and repeatable.  That seems to be not quite 
physically possible.  Could the oscillations be the pseudo turbulence reported by Chang and Liu (2000, Experiments in 
Fluids 29, 331-338)?  The authors should be able to tell if the figure is replotted with the units of the vertical axis 
replaced by pixels.
 }
Figure 12 shows results 120 in-beach, meaning that it is after breaking. However, such oscillation are conceivable
also without breaking; they may be due to infection point instability in the retarded boundary layer (see Pedersen et al 2013), but one would then expect a growth. The discussion of this around 203 is now elaborated. 
 In the paper by Chang and Liu, they see an bias error in PIV algoritm, which result in Psedo turbulence intensities in the non breaking waves. They sugguest that the error is related to the ratio (1:6 in their case) of the particle image to the pixel size in the images. This is often referred to as peak locking in PIV. However the oscillations in Figure 12, is found by  performing PTV on the images. Moreover, our particles range from ... to ... pixels in size and
our results are thus not prone to peak-locking.
\todo{Beskrevet partikkel/pixel rate i tekst ? Improve text around 203}
