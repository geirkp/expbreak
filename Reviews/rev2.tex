
\documentclass[11pt]{article}
\usepackage{epsfig,epsf,rotating,graphics,colordvi,color,a4}
\textwidth=16cm
\textheight=22cm
\begin{document}

\newcommand{\refpoint}[1]{\ \vspace{0.3cm}\\ {\em #1}\  \vspace{0.3cm}\\ }
\newcommand{\todo}[1]{\ \\ {\bf To do: #1}}

Reviewer \#2
\\
We are grateful to reviewer for his positive attitude and valuable suggestions.
\refpoint{First of all, the title is somewhat misleading. As mentioned in the beginning of this report, it deals also with 
numerical aspect and non-breaking waves. Could authors consider finding a more suitable title?}
Agreed.
The new title is ``Investigation of breaking and non-breaking solitary waves and measurements of swash zone dynamics on 
a $5^\circ$ beach.''

\refpoint{Then, there are two main issues with the experiments: 

1. One is the effects of deformed beach as discussed in the last paragraph on page 6. If the systematic depression along the centreline is significant enough to cause the pronounced transverse variation of runup, authors may also have to report runup value averaged over the transverse direction as well as the maximum value.}
For the non breaking wave $\alpha =0.10$, a traverse field of view average runup height is added in the manuscript (page 7, line 142). In the breaking wave cases the shoreline deformation was too large, such that only pieces of the shoreline was captured in the field of views. An average of the captured shoreline would therefore be misleading and are not included. 
\refpoint{2. The other is the uncertainty of the data. In many parts of the manuscript, authors implied that the uncertainty of the velocity data are significant, which results in marked variations among different runs in figures 10 \& 11, for example.}
The uncertainty related to the data, the experimental setup, and the PIV algorithm can be related to the deviation between the runs for the non breaking case. The deviation in Figure 9a is calculated to 0.16 cm/s, and is reported in the text (page 9, line 178). The deviation between the runs observed in Figure 10 and 11 is larger than the ones in Figure 9a, and the deviations must therefore be interpreted as physical.
\refpoint{ It is understandable given that the challenging situation with bubbles in thin layer with high velocity flow. However, there is no formal discussion on the uncertainty other than the discrepancy between the runs. Authors should carry out formal uncertainty analysis and quantify it.}
An analysis of the errors and the uncertainty associated with the measurement technique PIV are added to the manuscript at the beginning of page 4.

\refpoint{Additionally, there are some minor points:
(1) Second from the last sentence in abstract: plunger breaker $->$ plunging breaker
(2) First paragraph on page 3 starting with 'Nominally' is not a complete sentence. Also, /alpha is not previously defined. 
(3) First paragraph of Section 3 on page 5: plunger breaker -> plunging breaker
(4) Second paragraph on page 11: origo -> origin
(5) The sentence just before Section 4 on page 14: Extended bubbly region could OR could NOT have affected the irregular runup for large /alpha. The sentence is a just pure speculation and better be dropped.}

All the minor points are corrected.

\end{document}
