The event of a swash tongue moving with a constant velocity U upward a beach, can be compared to the problem of accelerating an infinite long plate from rest to a constant velocity U with a viscous fluid on top, Stokes first problem \citep{white2006viscous}. Dimensional analysis will then give us a relationship between the boundary layer thickness $\delta $ and the viscosity $\nu$ and the time $t$,  $\delta \approx \sqrt{\nu t}$.  This implies that the boundary layer will grow with time, and this seems to be the case for the small non-breaking waves with amplitude  $A/H=0.0989$ (Figure \ref{fig:BIM3_tim}). However, for the strong plunging breakers the boundary layer decreases with time. This implies that the motion is more  irregular than for the non-breaking waves. 

%In addition, is the boundary layer thicker for the breaking waves. 
% For $A/H=0.2958$ to $A/H=0.4874$ the outer flow seems to start at higher values of z than for the non-breaking cases, this may imply that viscous effects has a larger influence on the breaking cases. For $A/H=0.0989$ to $A/H=0.1981$ the flow can be divided into three regions, the upper region (outer flow) where $\frac{du}{dz}=0$, the middle region where $\frac{du}{dz}>0$ and the region closest to the beach wall, where $\frac{du}{dz}<0$. 
% For breaking cases $A/H=0.2958-0.4874$, it seemed that the region closest to the wall where $\frac{du}{dz}<0$ is almost non existent for the velocity profiles obtained before flow reversal. This can be interpreted as irregular motion. The section where $\frac{du}{dz}<0$, reoccurred for the velocity profiles obtained after flow reversal for cases  $A/H=0.2958-0.4874$. This indicated that the deceleration of the field makes the motion more regular.

The first part of the evolution the boundary layer
resembles Stoke's first problem (see, for instance, 
\citep{white2006viscous}), where a flat plate is set in instantaneou motion below a fluid at rest. The boundary layer thickness will then
grow in proprtion to  $ \sqrt{\nu t}$. 

The first part of the evolution the boundary layer
resembles Stoke's first problem (see, for instance, 
\citep{white2006viscous}), where a flat plate is set in instantaneou motion below a fluid at rest. The boundary layer thickness will then
grow in proprtion to  $ \sqrt{\nu t}$. 
   For the lowest amplitude, $A/H=0.0989$, this flow
compares This implies that the boundary layer will grow with time, and this seems to be the case for the small non-breaking waves with amplitude  $A/H=0.0989$ (Figure \ref{fig:BIM3_tim}). However, for the strong plunging breakers the boundary layer decreases with time. This implies that the motion is more  irregular than for the non-breaking waves. 

%The smallest cases with $A/H \approx 0.10, 0.12 $ did not break while propagating on a beach with inclination $5.1^\circ$ . All the other waves developed into plunging breakers. Surface elevation of the incident waves were measured with ultrasonic gauges, and coincided with theory from \cite{tanaka1986stability}. PIV  were performed on images captured from three different FOVs. The FOVs were located from 40cm to 120cm from where the still water level intersected with the beach. 



The experimental result from  the non-breaking  waves generated in this study coincide with the numerical result from BIM model. Both, the surface elevation and velocities seems to agree with numerical result. However, deviations between the computed and the measured maximum runup  was observed for the smallest waves. Also the BIM model over predict the velocity in front of the wave. Discrepancies between computation and measurement may be due to viscosity effect in the thin swash tongue, but may also be caused by bending effects of the beach. The beach bended due to its own weight and an additional bending was also observed due to wave load. 

